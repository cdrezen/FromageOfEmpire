\documentclass{article}

\title{Projet - Approche objet}
\author{Kerim Canakci (22018932), Corentin Drezen (22309435)}
\date{Décembre 2023}

\begin{document}

\maketitle

\section*{Analyse}

On doit concevoir un jeu qui doit permettre à un joueur de gérer des \textbf{Resource}, des \textbf{habitant}s et des \textbf{Building}s, tout en cherchant à maximiser leur économie.
\\
\par Une classe \textbf{Resource} doit permettre de représenter les différentes ressources disponibles dans le jeu (\textit{Gold, Food, Wood, etc}).
\begin{itemize}
    \item Comment faire en sorte qu’une \textbf{Resource} puisse représenter différentes sortes de ressources ?
    \item Comment doit-on gérer leurs quantités ?
\end{itemize}
\ \
\par Une classe \textbf{Building} pour représenter les bâtiments. Ils doivent contenir un nombre prédéfini d’\textbf{habitant}s et/ou de \textbf{travailleur}s. Un \textbf{Building} \underline{peut} produire et/ou consommer des \textbf{Resource}s proportionnellement au nombre de travailleurs. Il doit être construit en une durée avec des \textbf{Resource}s.
\begin{itemize}
    \item Comment gérer la différence entre un \textbf{Building} “usine”, “logement” ou “logement d’usine” ? Comment différencier les \textbf{travailleur}s des \textbf{habitant}s si une “usine” est aussi “logement” ? 
    \item Comment gérer la production/consommation de \textbf{Resource} suivant le \textbf{Building} ? 
    \item Comment doit-on faire progresser la construction ?
\end{itemize}
\ \
\par Des \textbf{habitant}s \underline{peuvent} être \textbf{travailleur}s et doivent être logés dans des \textbf{Building} “logement” ou “logement d’usine”. Ils doivent consommer une \textbf{Resource} “Food” par unité de temps. 
\begin{itemize}
    \item Un \textbf{habitant} peut-il être simplement un entier ? A-t-on besoin de savoir si un \textbf{habitant} est \textbf{travailleur} et s'il est logé ?
\end{itemize}
\ \
\par Une classe \textbf{Manager} doit orchestrer la consommation/production des \textbf{Building}s et \textbf{habitant}s et permettre aux joueur de construire ou détruire des \textbf{Building}s, d'y ajouter ou retirer des \textbf{travailleur}s.
\begin{itemize}
    \item Comment doit-on représenter les \textbf{Resource}s, \textbf{Building}s et \textbf{habitant}s pour qu’ils soient facilement modifiables par cet objet ? 
    \item Comment faire en sorte que l’utilisateur puisse intervenir dans cette gestion ?
\end{itemize}
\ \
\par Il doit y avoir une \textbf{interface utilisateur} qui permette à l'utilisateur d’intervenir sur \textbf{Manager}. Chaque saisie au clavier doit correspondre à une unité de temps ou une action. 
\begin{itemize}
    \item Comment interpréter les différentes commandes entrées par l’utilisateur ?
\end{itemize}
\ \
Nous devons donc trouver une architecture qui convienne aux différents objets “\textbf{Resource}”, “\textbf{Building}”, “\textbf{habitant}”, et “jeu” et qui respecte l’approche objet.

\section*{Conception}

\end{document}
